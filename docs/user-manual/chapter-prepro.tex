
\section{Using a simple preprocessor with {\tt -pp}}

Here is a typical example of using an installed preprocessor:

\begin{verbatim}
begin library "my_lib"
  files = [ "foo.ml" (pp = "camlp4o") ]
  requires = [ "camlp4o" ]
end
\end{verbatim}

Note that the \verb+camlp4o+ package should be defined as a program,
not a library to avoid linking.

\subsection{Using a preprocessor built in the same project}

Here is an example where the preprocessor is built in the same project:

\begin{verbatim}
begin library "my_lib"
  ocp_pp =  "%{ocp-pp_FULL_DST_DIR}%/ocp-pp.byte"
  files = [ "foo.ml" (pp = ocp_pp)
            "bar.ml" ]
  requires = [ "ocp-pp" ]
  pp_requires = [ "ocp-pp:byte" ]
end
\end{verbatim}

or alternatively:

\begin{verbatim}
begin library "my_lib"
  ocp_pp =  "%{ocp-pp_FULL_DST_DIR}%/ocp-pp.byte"
  files = [ "foo.ml" (pp = ocp_pp
                      pp_deps = [ ocp_pp ]  )
            "bar.ml"
  ]
  requires = [ "ocp-pp" ]
end
\end{verbatim}

The name of the preprocessor \verb|%{ocp-pp_FULL_DST_DIR}%/ocp-pp.byte|:
tells \ocpbuild{} that it should find the preprocessor as \verb|ocp-pp.byte|
in the result directory of \verb|ocp-pp|. The package of the preprocessor
itself should appear in the \verb|requires|.

The other options are:
\begin{itemize}
\item \verb|pp| builds the content of the \verb|-pp| option of OCaml. It
  is either a string with the executable to be used, or a list of strings,
  for the executable and its arguments, i.e. \verb|"prepro option"| will
  trigger an error and should be replaced by \verb|[ "prepro" "option"|.
\item \verb|pp_deps| is a list of files that should be built before running
  the preprocessor. Note that you should use \verb|%{package_FULL_DST_DIR}%|
  substitutions if the dependencies are built in the same project.
\item \verb|pp_requires| is a list of dependencies, but specified as
  package outputs, i.e. \verb|"package:kind"| where \verb|kind| is either
  \verb|byte| or \verb|asm|.
\end{itemize}

These options can be specified either at the package level (they will apply
to all files) or at the file level.

Here is another example, where the interface file \verb|foo.mli| is
specified separately to prevent the preprocessor from being called on it:
\begin{verbatim}
begin library "my_lib"
  ocp_pp =  "%{ocp-pp_FULL_DST_DIR}%/ocp-pp.byte"
  files = [ "foo.mli"
            "foo.ml" (pp = ocp_pp
                      pp_deps = [ ocp_pp ]  )
            "bar.ml"
  ]
  requires = [ "ocp-pp" ]
end
\end{verbatim}
 
\section{Defining a syntax package}

It is also possible to define a package as a \verb|syntax|, so that
it can be used in other packages.

First, let's define a syntax package for the \verb|js_of_ocaml| Camlp4 syntax:

\begin{verbatim}
begin (* begin..end to limit the scope of generated=true *)
  generated = true   (* the following packages are already installed *)
  (* the following packages are installed in this directory: *)
  dirname = [ "%{js_of_ocaml.syntax_SRC_DIR}%" ]
  begin objects "js_of_ocaml.camlp4-syntax.objects"
    files = [ "pa_js.ml" ] (* we want ps_js.cmo to be linked by camlp4o *)
  end
  begin syntax (* this package is a "syntax" *)
      "js_of_ocaml.camlp4-syntax" (* the name used for that syntax *)
    (* we need "camlp4o" to be used, just with the previous package *)
    requires = [ "js_of_ocaml.camlp4-syntax.objects" "camlp4o" ]
  end
end
\end{verbatim}

Now, we can use it in one of our packages:

\begin{verbatim}
begin library "my-jslib"
  has_asm = false (* no need to build a native version *)
  syntax = "js_of_ocaml.camlp4-syntax" (* the syntax to be used ! *)
  files = [ "utils.ml" ]
  (* we need both the syntax and the library: *)
  requires = [ "js_of_ocaml.camlp4-syntax" "js_of_ocaml" ]
end

\end{verbatim}

